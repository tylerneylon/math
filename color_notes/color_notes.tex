% Options for packages loaded elsewhere
\PassOptionsToPackage{unicode}{hyperref}
\PassOptionsToPackage{hyphens}{url}
\PassOptionsToPackage{dvipsnames,svgnames,x11names}{xcolor}
%
\documentclass[
]{article}
\usepackage{amsmath,amssymb}
\usepackage{fancyvrb}
\usepackage{fvextra}
%\RecustomVerbatimEnvironment{verbatim}{Verbatim}{commandchars=\\\{\}}
\usepackage{lmodern}
\usepackage{bold-extra}
\usepackage{iftex}
\ifPDFTeX
  \usepackage[T1]{fontenc}
  \usepackage[utf8]{inputenc}
  \usepackage{textcomp} % provide euro and other symbols
\else % if luatex or xetex
  \usepackage{unicode-math}
  \defaultfontfeatures{Scale=MatchLowercase}
  \defaultfontfeatures[\rmfamily]{Ligatures=TeX,Scale=1}
\fi
% Use upquote if available, for straight quotes in verbatim environments
\IfFileExists{upquote.sty}{\usepackage{upquote}}{}
\IfFileExists{microtype.sty}{% use microtype if available
  \usepackage[]{microtype}
  \UseMicrotypeSet[protrusion]{basicmath} % disable protrusion for tt fonts
}{}
\makeatletter
\@ifundefined{KOMAClassName}{% if non-KOMA class
  \IfFileExists{parskip.sty}{%
    \usepackage{parskip}
  }{% else
    \setlength{\parindent}{0pt}
    \setlength{\parskip}{6pt plus 2pt minus 1pt}}
}{% if KOMA class
  \KOMAoptions{parskip=half}}
\makeatother
\usepackage{xcolor}
\setlength{\emergencystretch}{3em} % prevent overfull lines
\providecommand{\tightlist}{%
  \setlength{\itemsep}{0pt}\setlength{\parskip}{0pt}}
\setcounter{secnumdepth}{5}
\makeatletter
\@ifpackageloaded{subfig}{}{\usepackage{subfig}}
\@ifpackageloaded{caption}{}{\usepackage{caption}}
\captionsetup[subfloat]{margin=0.5em}
\AtBeginDocument{%
\renewcommand*\figurename{Figure}
\renewcommand*\tablename{Table}
}
\AtBeginDocument{%
\renewcommand*\listfigurename{List of Figures}
\renewcommand*\listtablename{List of Tables}
}
\newcounter{pandoccrossref@subfigures@footnote@counter}
\newenvironment{pandoccrossrefsubfigures}{%
\setcounter{pandoccrossref@subfigures@footnote@counter}{0}
\begin{figure}\centering%
\gdef\global@pandoccrossref@subfigures@footnotes{}%
\DeclareRobustCommand{\footnote}[1]{\footnotemark%
\stepcounter{pandoccrossref@subfigures@footnote@counter}%
\ifx\global@pandoccrossref@subfigures@footnotes\empty%
\gdef\global@pandoccrossref@subfigures@footnotes{{##1}}%
\else%
\g@addto@macro\global@pandoccrossref@subfigures@footnotes{, {##1}}%
\fi}}%
{\end{figure}%
\addtocounter{footnote}{-\value{pandoccrossref@subfigures@footnote@counter}}
\@for\f:=\global@pandoccrossref@subfigures@footnotes\do{\stepcounter{footnote}\footnotetext{\f}}%
\gdef\global@pandoccrossref@subfigures@footnotes{}}
\@ifpackageloaded{float}{}{\usepackage{float}}
\floatstyle{ruled}
\@ifundefined{c@chapter}{\newfloat{codelisting}{h}{lop}}{\newfloat{codelisting}{h}{lop}[chapter]}
\floatname{codelisting}{Listing}
\newcommand*\listoflistings{\listof{codelisting}{List of Listings}}
\makeatother
\ifLuaTeX
  \usepackage{selnolig}  % disable illegal ligatures
\fi
\IfFileExists{bookmark.sty}{\usepackage{bookmark}}{\usepackage{hyperref}}
\IfFileExists{xurl.sty}{\usepackage{xurl}}{} % add URL line breaks if available
\urlstyle{same} % disable monospaced font for URLs
\hypersetup{
  pdftitle={Notes on Color},
  pdfauthor={Tyler Neylon},
  colorlinks=true,
  linkcolor={black},
  filecolor={Maroon},
  citecolor={Blue},
  urlcolor={Blue},
  pdfcreator={LaTeX via pandoc}}

\title{Notes on Color}
\author{Tyler Neylon}
\date{\href{https://tylerneylon.com/a/7date/}{352.2024}}

%%%%%%%%%%%%%%%%%%%%%%%%%%%%%%%%%%%%%%%%%%%%%%%%%%%%%%%%%%%%%%%%%%%%%%%%%%%
% Begin custom, non-pandoc commands.

\newcommand{\customstrut}{\rule[-3mm]{0mm}{7.5mm}}
\newenvironment{densearray}{\begin{array}{rcl}}{\end{array}}
\newcommand{\class}[1]{}
\newcommand{\Rule}[3]{}
\newcommand{\optquad}{\quad}
\newcommand{\smallscrneg}{}
\newcommand{\smallscr}[1]{}
\newcommand{\bigscr}[1]{#1}
\newcommand{\smallscrskip}[1]{}

% I learned some things from these two links:
% https://tex.stackexchange.com/questions/145812/using-fbox-in-a-newenvironment
% https://tex.stackexchange.com/questions/120042/splitting-a-command-syntax-across-a-newenvironment-definition

\newsavebox{\mybox}
\newenvironment{myboxed}{\begin{lrbox}{\mybox}\begin{minipage}{0.98\textwidth}}{\end{minipage}\end{lrbox}\fbox{\usebox{\mybox}}}

\newcommand{\boxedstart}{\begin{myboxed}}
\newcommand{\boxedend}{\end{myboxed}}

\newcommand{\crossedouty}{\dot y \kern -4.5pt \raise 4.9pt \hbox{\(\scriptscriptstyle\diagup\)}}
\newcommand{\crossedoutone}{\dot 1 \kern -5.1pt \raise 6.6pt \hbox{\(\scriptscriptstyle\diagup\)}}
\newcommand{\crossedouttwo}{\dot 2 \kern -5.1pt \raise 6.6pt \hbox{\(\scriptscriptstyle\diagup\)}}
\newcommand{\crossedoutthree}{\dot 3 \kern -5.1pt \raise 6.6pt \hbox{\(\scriptscriptstyle\diagup\)}}
\newcommand{\crossedoutfive}{\dot 5 \kern -5.1pt \raise 6.6pt \hbox{\(\scriptscriptstyle\diagup\)}}
\newcommand{\crossedoutsix}{\dot 6 \kern -5.1pt \raise 6.6pt \hbox{\(\scriptscriptstyle\diagup\)}}
\newcommand{\crossedoutseven}{\dot 7 \kern -5.1pt \raise 6.6pt \hbox{\(\scriptscriptstyle\diagup\)}}
\newcommand{\lowerhaty}{\lower 1ex\hbox{\(\hat y\)}}
\newcommand{\lhy}{\lower 1ex\hbox{\(\hat y\)}}

\let\smallstart\iffalse
\let\smallend\fi

% End custom, non-pandoc commands.
%%%%%%%%%%%%%%%%%%%%%%%%%%%%%%%%%%%%%%%%%%%%%%%%%%%%%%%%%%%%%%%%%%%%%%%%%%%

\begin{document}
\maketitle

\newcommand{\R}{\mathbb{R}}
\newcommand{\N}{\mathbb{N}}
\newcommand{\eqnset}[1]{\left.\mbox{$#1$}\;\;\right\rbrace\class{postbrace}{ }}
\providecommand{\latexonlyrule}[3][]{}
\providecommand{\optquad}{\class{optquad}{}}
\providecommand{\smallscrneg}{\class{smallscrneg}{ }}
\providecommand{\bigscr}[1]{\class{bigscr}{#1}}
\providecommand{\smallscr}[1]{\class{smallscr}{#1}}
\providecommand{\smallscrskip}[1]{\class{smallscrskip}{\hskip #1}}

\newcommand{\mydots}{{\cdot}\kern -0.1pt{\cdot}\kern -0.1pt{\cdot}}

\newcommand{\?}{\stackrel{?}{=}}
\newcommand{\sign}{\textsf{sign}}
\newcommand{\order}{\textsf{order}}
\newcommand{\flips}{\textsf{flips}}
\newcommand{\samecycles}{\textsf{same$\\\_$cycles}}
\newcommand{\canon}{\textsf{canon}}
\newcommand{\cs}{\mathsf{cs}}
\newcommand{\dist}{\mathsf{dist}}
\renewcommand{\theenumi}{(\roman{enumi})}

{[} Formats:
\href{http://tylerneylon.com/a/color_notes/color_notes.html}{html}
\textbar{}
\href{http://tylerneylon.com/a/color_notes/color_notes.pdf}{pdf}
\(\,\){]}

This post captures my own notes about color, categorized according to
some key questions I had when I began.

I have a daydream of one day writing up something like an engineer's
guide to color that is at once engaging, well-designed, educational, and
a pleasure to read. For now I'm just aiming for educational.

\hypertarget{the-nature-of-human-vision}{%
\section{The nature of human vision}\label{the-nature-of-human-vision}}

\begin{itemize}
\tightlist
\item
  How do we know that color is a 3-dimensional thing?

  \begin{itemize}
  \tightlist
  \item
    Related: If we have 3 types of cones plus rods, why isn't color
    4-dimensional?
  \end{itemize}
\item
  Why are red, green, and blue special?
\item
  How do we know humans have 3 types of cones?
\item
  How do rods and cones perceive color?

  \begin{itemize}
  \tightlist
  \item
    Related: What is a mathematical model for how light in the world is
    translated into signals in the brain?
  \end{itemize}
\item
  How do we know the actual cone sensitivies?
\item
  What do negative coefficients mean in color matching experiments, and
  how can I think about those?
\item
  Why do the ends of the rainbow meet up so nicely?

  \begin{itemize}
  \tightlist
  \item
    Related: Are all the colors we see based on physically real colors?
  \end{itemize}
\item
  Why do we see the light spectrum that we do?
\item
  Why is the chromaticity diagram the shape that it is?
\item
  Are there colors we could theoretically perceive, but can't exist in
  reality?
\item
  How can I model, with mathematical precision, the common types of
  color blindness?
\end{itemize}

\hypertarget{questions-about-light-and-color-itself}{%
\section{Questions about light and color
itself}\label{questions-about-light-and-color-itself}}

\begin{itemize}
\tightlist
\item
  How do blacklights work?
\item
  How does infrared light interact with heat?
\end{itemize}

\hypertarget{reproducing-colors}{%
\section{Reproducing colors}\label{reproducing-colors}}

\begin{itemize}
\tightlist
\item
  How do monitors and printers reproduce any color?
\item
  Why can't a monitor display all possible colors?
\item
  What are color spaces?
\item
  What are the HSL and HSV color spaces, and how do they differ?
\item
  What's the difference between chroma and saturation?
\item
  What are the formulas for switching between RGB, HSL, and HSV?
\item
  What does the name sRGB indicate?
\end{itemize}

\end{document}

