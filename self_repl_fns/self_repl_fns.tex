\documentclass[]{article}
\usepackage{lmodern}
\usepackage{amssymb,amsmath}
\usepackage{ifxetex,ifluatex}
\usepackage{fixltx2e} % provides \textsubscript
\ifnum 0\ifxetex 1\fi\ifluatex 1\fi=0 % if pdftex
  \usepackage[T1]{fontenc}
  \usepackage[utf8]{inputenc}
\else % if luatex or xelatex
  \ifxetex
    \usepackage{mathspec}
    \usepackage{xltxtra,xunicode}
  \else
    \usepackage{fontspec}
  \fi
  \defaultfontfeatures{Mapping=tex-text,Scale=MatchLowercase}
  \newcommand{\euro}{€}
\fi
% use upquote if available, for straight quotes in verbatim environments
\IfFileExists{upquote.sty}{\usepackage{upquote}}{}
% use microtype if available
\IfFileExists{microtype.sty}{%
\usepackage{microtype}
\UseMicrotypeSet[protrusion]{basicmath} % disable protrusion for tt fonts
}{}
\usepackage{graphicx}
\makeatletter
\def\maxwidth{\ifdim\Gin@nat@width>\linewidth\linewidth\else\Gin@nat@width\fi}
\def\maxheight{\ifdim\Gin@nat@height>\textheight\textheight\else\Gin@nat@height\fi}
\makeatother
% Scale images if necessary, so that they will not overflow the page
% margins by default, and it is still possible to overwrite the defaults
% using explicit options in \includegraphics[width, height, ...]{}
\setkeys{Gin}{width=\maxwidth,height=\maxheight,keepaspectratio}
\ifxetex
  \usepackage[setpagesize=false, % page size defined by xetex
              unicode=false, % unicode breaks when used with xetex
              xetex]{hyperref}
\else
  \usepackage[unicode=true]{hyperref}
\fi
\hypersetup{breaklinks=true,
            bookmarks=true,
            pdfauthor={Tyler Neylon},
            pdftitle={Self-Replicating Functions},
            colorlinks=true,
            citecolor=blue,
            urlcolor=blue,
            linkcolor=black,
            pdfborder={0 0 0}}
\urlstyle{same}  % don't use monospace font for urls
\setlength{\parindent}{0pt}
\setlength{\parskip}{6pt plus 2pt minus 1pt}
\setlength{\emergencystretch}{3em}  % prevent overfull lines
\setcounter{secnumdepth}{5}

\title{Self-Replicating Functions}
\author{Tyler Neylon}
\date{204.2016}

\begin{document}
\maketitle

These are notes I'm creating for myself as I explore functions \(f\)
that can be written as a sum \(f = g_1 + g_2\) where \(g_1\) and \(g_2\)
are the same up to symmetry, and both \(g_1\) and \(g_2\) strongly
resemble the original function \(f\). When a function \(f\) has these
properties, I informally call it a \emph{self-replicating function}.

Like the word \emph{fractal}, this term is not rigorously defined --- in
particular, it depends on the ambiguous notion of ``strong resemblance''
--- although I plan to investigate more precise requirements below.

\section{Motivation}\label{motivation}

I became interested in self-replicating functions by working on
algorithms to procedurally generate 3d models of natural-looking trees.
When algorithmically making trees, it makes sense to start from the idea
of an \href{https://en.wikipedia.org/wiki/L-system}{\emph{L-system}},
which can be visualized as a kind of fractal in which a trunk forks into
branches that fork into smaller subranches, this process repeating
infinitely.

I noticed that tree-like \emph{L}-systems can have a large amount of
``branch overlap'' concentrated around a central area of their apparent
surface. For example, consider the two images below. On the left is a
standard \emph{L}-system along with a histogram showing the density of
leaf points along the edge. Intuitively, the leaf points achieve a
reasonable density even toward the extreme angles of the tree's top.
However, the density increases continuously toward the center.

We could think of each leaf point as doing a certain amount of work by
covering some area along the top of the \emph{L}-system. Each subtree is
so oblivious to its other subtrees that they overlap heavily, and the
central leaf points end up being highly redundant. To illustrate this
redundancy, the right-hand figure shows the exact same \emph{L}-system
with essentially half of the tree removed --- yet the shape formed by
the leaf points is only slightly changed.

\begin{figure}[htbp]
\centering
\includegraphics{images/pdfs/ellsystem2.pdf}
\caption{Left: An \emph{L}-system; Right: the same system with two large
subtrees removed. In both cases, a histogram of leaf point density is
provided around an outer ellipse.}
\end{figure}

One approach to smoothing out the distribution of leaf points would be
to compromise the fractal-like nature of the system by choosing each
line direction based on where it is within the fractal, rather than
simply by making each branching point a smaller version of its parent.
The line directions can be chosen so that the set of points at a fixed
distance from the trunk point form a set of equidistant angles from a
central point. The result is an extremely regular edge, as seen below.

\begin{figure}[htbp]
\centering
\includegraphics{images/pdfs/well_distributed_ell_like_system.pdf}
\caption{A \emph{L}-like system in which line directions are chosen to
maximize the regularity of leaf point distribution.}
\end{figure}

This is ideally efficient in that each leaf point is equally important
in forming the shape of the system. However, this system is defined in
terms of the path to each point. Is it possible to design a system so
that the overall distribution of leaf points is fairly even, yet each
subtree's shape is independent of its position within the full tree?

If this goal were achieved, we would necessarily have a leaf point
distribution which was the sum of two smaller versions of itself.
Intuitively, the leaf-point distribution of any \emph{L}-system is
already a self-replication function because, if its two main subtrees
have distribution functions \(g_1\) and \(g_2\), then the full tree has
distribution function \(f = g_1 + g_2\). I have to say
\emph{intuitively} here because I haven't formally defined the
leaf-point distribution of an \emph{L}-system.

Thus, \emph{L}-systems naturally coincide with self-replicating
functions. Although there are probably self-replicating functions which
do not correspond with \emph{L}-systems, I nonetheless find it
interesting to independently explore the world of self-replicating
functions.

\section{Piece-wise linear cases}\label{piece-wise-linear-cases}

\section{The normal curve}\label{the-normal-curve}

The normal curve is described by \(y = e^{-x^2/2}\).

\begin{figure}[htbp]
\centering
\includegraphics{images/normal@2x.png}
\caption{\(y=e^{-x^2/2}\)}
\end{figure}

\section{\texorpdfstring{Leaf-point distributions of
\emph{L}-systems}{Leaf-point distributions of L-systems}}\label{leaf-point-distributions-of-l-systems}

\section{Temporary example content}\label{temporary-example-content}

\textbf{Lemma 1}~ \emph{Content of lemma 1, including some \(\pi+3\)
mathy bits.}

\subsection{Subheader}\label{subheader}

Content

See my notes on Raney's lemmas for more examples.

Here is a reference (Knuth, Patashnik, and Graham 1998).

\section{Questions}\label{questions}

\begin{itemize}
\itemsep1pt\parskip0pt\parsep0pt
\item
  The ellipse around my first \emph{L}-system appears to fit
  surprisingly well. Is there a nice way to discover when an ellipse and
  an \emph{L}-system may fit like this? Is there, perhaps, a series of
  shapes which converge on the system or its leaf points, analogous to
  \href{http://mathworld.wolfram.com/MandelbrotSetLemniscate.html}{Mandelbrot
  set lemniscates}?
\item
  The histogram around my first \emph{L}-system appears simple in shape.
  Can its shape be described precisely?
\end{itemize}

\section*{References}\label{references}
\addcontentsline{toc}{section}{References}

Knuth, Donald E., Oren Patashnik, and Ronald L. Graham. 1998.
\emph{Concrete Mathematics: A Foundation for Computer Science}.
addison-wesley.

\end{document}
